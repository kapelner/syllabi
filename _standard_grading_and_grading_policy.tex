\section*{Grading and Grading Policy}\label{sec:grading}

Your course grade will be calculated based on the percentages as follows: 

\begin{table}[h]
\centering
\begin{tabular}{l|l}
Homework & 10\% \\
Midterm Examination I & 23\%\\
Midterm Examination II & 23\%\\
Final Examination & 39\% \\
Class participation & 5\%
\end{tabular}
\end{table}
\FloatBarrier

The semester is split into three periods :

\begin{enumerate}
\item From the beginning until midterm I. Midterm I covers material during this time.
\item From midterm I to midterm II. Midterm II covers material in this period only. 
\item From midterm II until the final. The final is cumulative and covers all course material.
\end{enumerate}

Each of the periods is assessed evenly. Thus, each period must count the same towards your grade. Since there is 75\% of the grade allotted to exams, there is 25\% allotted to each period. Thus, the final is upweighted towards the material covered in the third period. In summary, the final will have 5/35 points $\approx$ 14\% for the first period's material, 5/35 points $\approx$ 14\% for the second period's material and 25/35 points $\approx$ 71\% for the last period's material. A good strategy for the final is to just study the material after Midterm II and minimal studying for the previous material.