\subsection*{AI Policy}

ChatGPT, Gemini, perplexity, grok et al. (collectively \qu{AI}) is \emph{not required} for this course. But we must all acknowledge that it is now an integral (and soon to be ubiquitous) part of the world we live in. Most germane is the fact that sometime in 2024, it reached the point where it can solve the at-home exercises in this course. 

On the one hand, this presents a monumental pedagogical challenge. As mentioned above, the exercises are the only way to truly learn the concepts. If you use AI to solve the questions on autopilot, you will never master the concepts. And the exams will subsequently crush you. On the other hand, the educated citizens of the future need to learn to leverage AI to avoid being left behind. But here's my loose policy for those who wish to use AI for the at-home exercises:

\begin{itemize}
\item First, try your best to do the HW exercise yourself. 
\item If you \ingreen{can do the exercise}, treat the AI as a teaching assistant. Feel free to then show your answer to the AI to see if you missed any steps.
\item If you \inred{cannot do the exercise} after \emph{honestly trying}, treat the AI once again as a teaching assistant who is allowed to help you. Ask it for your missing step and \emph{tell it explicitly not to give you the answer}. Then the next step, and the next, etc. As for its response(s), paraphrase them onto your paper \emph{in your own words} and in the style, notation and dialect we use in this class. This paraphrasing exercise will be effective in helping you learn.
\item If you used AI at all for the exercise, be transparent and cite which specify AI model (along with date and version).
\item If you \emph{directly} copy from the AI and you cite it, you will still receive a zero for that exercise. If you directly copy from the AI and you \emph{don't cite the AI}, you will be penalized as this will be considered plagiarism. We can figure out if you copy (as we ourselves use AI daily).
\end{itemize}