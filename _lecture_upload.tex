\subsection*{Lecture Upload}\label{subsec:lecture_upload}

As many previous students have noted, my handwritten notes are useful to me and not to many others. (Although this has been improved by zoom where I now \qu{type} on the \qu{chalkboard}). I will still be rewarding students for taking notes, scanning them in and sending them to me. You will be rewarded in two ways: (1) if you do this for more than 10 lectures, you will be given the automatic 5 participation points (see grading policy on page \pageref{sec:grading}) for your classroom participation grade and (2) you have the option for me to say your name publicly on the \coursewebpagelink. Make sure you follow these instructions:

\begin{itemize}
\item You have \emph{one week only} from the time of the lecture to provide me lecture notes.
\item There must be \emph{one} file and it must be in PDF format only. 
\item The file must be $<$2MB. No exceptions. I will tell you to shrink the PDF if not.
\item The file must be named \texttt{lecxxkapelner.pdf} where you replace \texttt{xx} with two digits corresponding to the lecture number i.e. 01, 02, 09, 10, \ldots, 23 and you replace \texttt{kapelner} with your last name in all lowercase letters. If your file is renamed incorrectly, I will tell you to rename it and send it back.
\item Vertically oriented (readable without rotating your head 90 degrees.).
\item The scan is sharp and not blurry! Blurry submissions will be rejected.
\item The lecture must be added as a pull request through github. Follow these steps:

\begin{enumerate}[1.]
\item Create your \url{github.com} account (if you don't have one already).
\item Go to the \coursewebpagelink. In the top right corner you'll see a \qu{fork} button. Click that. It will create your own copy of the course's homepage.
\item Follow the instructions on this \href{https://opensource.com/article/19/7/create-pull-request-github}{site}.
\item If you wish to upload more than one lecture, do not fork again! You'll have to pull the current changes from the course homepage. Follow the instructions on this \href{https://docs.github.com/en/free-pro-team@latest/github/collaborating-with-issues-and-pull-requests/syncing-a-fork}{site}. 
\end{enumerate}
\item The pull request must consist of your full name and the appropriate link be placed in the README file one space \emph{after} my link and everyone else's text link. The link you use will be to the PDF in your \texttt{lectures} folder on your repository e.g.

\begin{quotation}
\noindent\texttt{
* Lecture 1 video on slack [(Prof)](https://github.com/kapelner/
QC\_\coursedept\_\coursenumber\_\semester\_\the\year/blob/master/lectures/lec01kap.pdf) [(John Doe)](https://github.com/johndoe99/QC\_\coursedept\_\coursenumber\_\semester\_\the\year/blob/
master/lectures/lec01doe.pdf)}
\end{quotation}

This may take a few tries for you to get this right. But once you do, you'll be a real open source contributor!
\end{itemize}

\noindent If you add your notes, you are (1) agreeing to the \href{https://opensource.org/licenses/MIT}{MIT license} which means someone can freely copy your notes and even make money off of it (and not owe you a cent!) and (2) since github is mirrored, once your upload is on the web, it is there indeliby forever. \inred{If you are not comfortable with these two points, do not send me your notes!!}
