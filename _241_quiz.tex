\section*{Syllabus Quiz}\label{quiz}

\begin{enumerate}[1.]
\item Is this a typical math course?

\begin{enumerate}[(a)]
\item Yes --- this is just another variation on calculus
\item No --- this is an applied math course and we will be discussing philosophy, decision theory and much else
\end{enumerate}

\item Is there a required textbook?

\begin{enumerate}[(a)]
\item Yes --- the Sheldon Ross book
\item No --- the book is not necessary to read, I can just use the class notes and listen to the lectures online
\end{enumerate}

\item Is there a required calculator?

\begin{enumerate}[(a)]
\item Yes --- the TI-83+
\item No --- but you have to bring a graphing calculator for exams
\item No --- but you need to bring some type of calculator for exams
\item No --- we will make no use of calculators in this course even on exams
\end{enumerate}

\item How many new material lectures are there?

\begin{enumerate}[(a)]
\item 27
\item 25
\item 23
\end{enumerate}


\item If I miss class, what do I do?

\begin{enumerate}[(a)]
\item Watch the online videos
\item Listen to online lectures only
\item Listen to online lectures and copy a friend's notes
\end{enumerate}

\item How can I be guaranteed the five points of classroom participation?

\begin{enumerate}[(a)]
\item Come to class every period
\item Scan 13 days of lecture notes is as PDFs (less than 2MB)
\item Scan one day of lecture notes as a JPG
\end{enumerate}

\item Can I work together with other students on the homework?

\begin{enumerate}[(a)]
\item Yes and we can collaborate handing in one writeup with all of our names on it
\item Yes and we can collaborate handing in separate writeups
\item No
\end{enumerate}

\item How do I earn extra points on the homework?

\begin{enumerate}[(a)]
\item Using \LaTeX ~and/or doing the extra credit problems. No sharing \LaTeX~code.
\item Only using \LaTeX . No sharing \LaTeX~code.
\item Only doing the extra credit problems
\item There is no way to earn extra points
\end{enumerate}

\item Can homework be handed in late?

\begin{enumerate}[(a)]
\item No. It will be given a zero.
\item Yes. Up to 3 days.
\item Yes. Up until the last lecture day of the semester.
\end{enumerate}


\item What are the exams based on?

\begin{enumerate}[(a)]
\item Skills built from doing homework problems
\item Details from lectures
\item Lecture topics that were not assessed in the homework assignments
\end{enumerate}

\item How are the grades computed?

\begin{enumerate}[(a)]
\item They are based on raw scores based on the percentages found in the course policy section of this document
\item They are based only on exams
\item They are mostly a reflection of your homework score
\item They are mostly a reflection of your classroom participation
\end{enumerate}


\item Does it matter if the exams are \qu{too hard}?

\begin{enumerate}[(a)]
\item Yes, then a lot of students won't get a high grade
\item No, because most students fail this course regardless of exam grades
\item No, not one bit; since the course is curved, we are graded on performance relative to others so the raw exam scores do not actually matter
\end{enumerate}

\item What happens when the exam is too easy? 

\begin{enumerate}[(a)]
\item Nothing! We'll all get high grades!
\item This is good thing: the curve will have more A's.
\item This is a bad thing: too many students ceiling out at 100 and the ones who make careless errors get 92's and may get B's unfairly.
\end{enumerate}

\item Given the answer to question 11, if you were to design an exam for this class, which is curved, what is the \textit{fairest} exam average?

\begin{enumerate}[(a)]
\item 20\%
\item 60\%
\item 90\%
\end{enumerate}

\item What can I use during the exams?

\begin{enumerate}[(a)]
\item My notes
\item My phone
\item My calculator
\item A cheat sheet and the textbook
\end{enumerate}

\item Missing the final will guarantee you get a \_\_\_ for your final grade.

\begin{enumerate}[(a)]
\item F
\item C-
\item WU
\item D
\end{enumerate}

\item How many students will get A's or A-'s?

\begin{enumerate}[(a)]
\item 5 students only regardless of enrollment
\item The top 10 students only regardless of enrollment
\item The top 50\% of students regardless of enrollment
\item The top $\approx$15-19\% of students regardless of enrollment
\end{enumerate}


\end{enumerate}