\subsection*{\LaTeX~Homework Bonus Points}

Part of good mathematics is its beautiful presentation. Thus, \ingreen{there will be a 1--7 point bonus} added to your theory homework grade  for typing up your homework using the \LaTeX ~typesetting system based on the elegance of your presentation. The bonus is arbitrarily determined by me.

I recommend using \href{http://overleaf.com}{overleaf} to write up your homeworks (make sure you upload both the hw\#.tex and the preamble.tex file). This has the advantage of (a) not having to install anything on your computer and thus not having to maintain your \LaTeX ~installation (b) allowing easy collaboration with others (c) alway having a backup of your work since it's always on the cloud. If you insist to have \LaTeX ~running on your computer, you can download it for Windows \href{http://www.miktex.org/download}{here} and for MAC \href{http://www.tug.org/mactex/}{here}. For editing and producing PDF's, I recommend \TeX works which can be downloaded \href{http://www.tug.org/texworks/#Getting_TeXworks}{here}. Please use the \LaTeX ~code provided on the \coursewebpagelink ~for each homework assignment. 

If you are handing in homework this way, read the comments in the code; there are two lines to comment out and you should replace my name with yours and write your section. The easiest way to use overleaf is to copy the raw text from hwxx.tex and preamble.tex into two new overleaf tex files with the same name. If you are asked to make drawings, you can take a picture of your handwritten drawing and insert them as figures or leave space using the \qu{$\backslash$vspace} command and draw them in after printing or attach them stapled.

Since this is extra credit, do not ask me for help in setting up your computer with \LaTeX~ in class or in office hours. Also, \textbf{never share your \LaTeX~code with other students} --- it is cheating and if you are found I will take it seriously.