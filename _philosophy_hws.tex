\subsection*{Philosophy of Homework}


Homework is the \textit{most} important part of this course.\footnote{In one student's \href{http://www.ratemyprofessors.com/ShowRatings.jsp?tid=1951051}{observation}, I give a \qu{mind-blowing homework} every week.} Success in Statistics and Mathematics courses comes from experience in working with and thinking about the concepts. It's kind of like weightlifting; you have to lift weights to build muscles. My job as an instructor is to provide assistance through your \href{http://en.wikipedia.org/wiki/Zone_of_proximal_development}{zone of proximal development}. With me, you can grow more than you can alone. To this effect, homework problems are color coded \ingreen{green} for easy, \inyellow{yellow} for harder, \inred{red} for challenging and \inpurple{purple} for extra credit. You need to know how to do all the greens by yourself. If you've been to class and took notes, they are a joke. Yellows and reds: feel free to work with others. Only do extra credits if you have already finished the assignment. The \qu{[Optional]} problems are for extra practice --- highly recommended for exam study.